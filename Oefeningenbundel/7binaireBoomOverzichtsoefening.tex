% !TEX root = B&G_oefeningen.tex
\chapter{Bomen: Overzichtsoefening}

In deze oefening krijg je een klasse \verb/Vak/ die sorteerbaar is op studiepunten en vervolgens op naam. Het vak `BOP' komt bijvoorbeeld voor het vak `OOP' en na het vak `Web1' omwille van verschil  in studiepunten. Het vak `Algo' komt voor het vak `Web2' omdat ze beide evenveel studiepunten hebben maar `Algo' alfabetisch voor `Web2' komt.

\subsection*{Doel van deze oefening}
Je hebt verschillende boomstructuren leren opzetten. Elke structuur heeft zijn kenmerken, i.h.b.\ wat zoeken naar grootste/kleinste waarde betreft. In deze oefening word je gevraagd na te denken welke structuur het beste past in de gegeven situatie.

We maken in deze oefening een brug naar de opleidingsonderdelen BOP en Web 2, in het bijzonder naar de  lijst- en db-klassen die daar in bod komen. We laten je inzien dat de klassen  \verb/BinarySearchTree/ uit labo 4  en \verb/BinaryMinHeap/ uit labo 6 gelijkaardige db-klassen zijn. 

\newpage
\subsection*{Studeren naar gelang het aantal studiepunten}
De examens naderen, je hebt nog veel studeerwerk. Je plant om zoveel mogelijk vakken te studeren en studeert dus eerst de vakken met het minst aantal studiepunten.
\begin{enumerate}
\item Welke datastructuur gebruik je om de vakken op te slaan zodat je
\begin{itemize}
\item  snel het vak  met het minst aantal studiepunten  terugvindt
\item snel dat vak kan verwijderen
\item snel opnieuw het vak met het minst aantal studiepunten terugvindt
\end{itemize}
\item Maak een nieuw Java Project `VakkenStuderen' aan. Organiseer de klassen nu zoals je leerde in de lessen BOP en Web2. 
\begin{enumerate}
\item Maak een package \verb/domain.model/ waar je de gegeven klasse \verb/Vak/ in plaatst.
\item Maak een package \verb/domain.db/ en de klasse \verb/VakkenLijst/. Deze klasse heeft als instantieveranderlijke de lijst met vakken, bewaard  in de door jou gekozen datastructuur. Waarschijnlijk heb je gekozen voor een boomstructuur. Kopieer de overeenstemmende methodes uit één van de voorgaande labo's naar de nieuwe klasse. De klasse die je nu aanmaakt is dus een lijst-klasse die je kent uit BOP of de db-klasse die je kent uit Web 2.
\item Maak een package \verb/ui/ aan die de user interface klasse bevat. Zet de gegeven klasse \verb/VakkenMain/ in deze package.  In dit project is de user interface een main methode die gegevens wegschrijft naar het scherm met het commando \verb/System.out.println()/. 
\end{enumerate}
\item Schrijf code in de main methode zodat de gegeven vakken opgeslagen/toegevoegd worden in een instantie van de db-klasse. 
\item \label{scherm} Schrijf naar het scherm welk vak je eerst gaat studeren (i.e.\ het vak met het minst aantal studiepunten). \\Voorbeeld van uitvoer: \\ \verb/Dit vak moet je eerst studeren: Algo met 3 studiepunten/.
\item Als je klaar bent met Algo, voeg je in de main een lijn code toe die dit vak uit de lijst met te studeren vakken verwijdert. 
\item Laat de db-klasse berekenen welk vak je vervolgens moet studeren. Schrijf dit naar het scherm zoals in oefening \ref{scherm}. 
\item Je bent vergeten dat je het vak Computersystemen (3 studiepunten) ook nog moet studeren. Voeg dit vak toe aan de lijst via de main-methode. 
\item Schrijf naar het scherm welk vak het meest aantal studiepunten heeft. 
\item ({\em moeilijker}) \label{lijstVakken} Schrijf alle resterende vakken uit waarbij de vakken vermeld worden in stijgende orde van aantal studiepunten. 
\item Maak een nieuwe db- en main-klasse aan. Kies nu voor een andere datastructuur waarbij je
\begin{itemize}
\item snel een overzicht van vakken kan gegeven, geordend naargelang het aantal studiepunten
\item zowel het vak met het grootste en het vak met het minste aantal studiepunten relatief snel kan vinden.
\end{itemize}
\item Laat de nieuwe main klasse naar het scherm schrijven welk vak respectievelijk minst en meest aantal studiepunten heeft.
\item Maak oefening \ref{lijstVakken} opnieuw. 
\end{enumerate}


